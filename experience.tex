%\section{Skills Spec}
%\cventry{}{Parallel programming}{}{}{}
%{All the works are done on the platform with a 12 core cpu.\newline I use (i) MPI to implement and optimize odd-even sort algorithm. (ii) pthread \& openmp to solve N-body problem with Barnes-Hut algorithm, and to generate mandelbrot set. (iii) CUDA to solve All-Pairs-Shortest-Path problem with 3-phase Floyd-Warshall and optimization}

%\section{Honors and Awards}
%\cvline{2014}{\textbf{National Tsing Hua University, Collegiate Programming Examination} {Qualified}}
%\cvline{2013}{\textbf{National Tsing Hua University, Academic Excellence Award}}

\section{Work Experiences}
\cvline{October 2022 - January 2023}{Research Assistant, Research Center for IT Innovation, Academia Sinica}
\cvline{September 2020 - June 2022}{Research Assistant, Studio of Computer Research on Music and Multimedia Lab, Computer Science Department, National Cheng Kung University}
% \cvline{February 2017 - June 2017}{Research Assistant, Large-scale System Architecture Lab, Computer Science Department, Nation Tsing Hua University}
% \cvline{February 2016 - June 2017}{Violinst, Tsing-Hua Orchestra}
% \cvline{August 2016}{Lecturer, Computer Science Camp for Freshmen, NTHU}
% \cvline{April 2015 - June 2016}{Research Assistant, Networking and Multimedia System Lab, Computer Science Department, Nation Tsing Hua University}
% \cvline{August 2015}{Lecturer, Computer Science Camp for Freshmen, NTHU}

%\section{Selected Coursework}
%\cvline{}{Computer Graphic (A+), Software Studio (A)}
%\cvline{}{Data Structures (A+), Computer Architecture (A)}

% \newpage
% \section{Research Experience}
% \cventry{}{Gamified Crowdsourcing System for Urban Computing}{}{}{}
% {Gamified Crowdsourcing System for Urban Computing is a porject of trying to solve the problem of taking advantage of crowdsourcing. We design a system containing the server and the mobile game named "Pokemon". We send requests to users, and users can complete the assignment by playing Pokemon game. Users will get reward in any type, such as achievement in the mobile game. For instance, server sends a request of taking a picture of the scene of the car accident to the certain user, and the user shoots it with the camera of Pokemon game and uploads the picture to the server, and then the user will get money in Pokemon game. This project consists of server side and client side. The server side has to solve the accountability problem, make sure requsts will be finished. The client side is the Pokemon game, used to solve the incentive problem. Users receive requests from server and make it with this game. In this project, I take responsibility for designing the Pokemon game}
% \section{Projects}
% \cventry{}{[1] Improve Spark Performance with Auto-checkpoint}{}{}{}
% {Using mobile devices to watch online multimedia content is getting increasingly popular all over the world. However, in developing countries, rural areas, or over-populated cities, many mobile users do not have Internet access. For instance, only 15\% of mobile users have mobile Internet access in Africa, because of weak or non-existing network infrastructure. Therefore, people in these areas have little chance to access online multimedia content, such as news, advertisements, and movies. That is, they suffer from digital divide from other parts of the world. We propose a Challenged Content Delivery Network (CCDN) for: (i) mobile users without stable Internet access to prefetch multimedia content, and (ii) service providers to deliver their multimedia content to more people. Our CCDN system consists of several entities: a distribution server, several local proxies, and many mobile clients. Local proxies are intelligent WiFi access points with storage spaces, which are deployed at crowded places. Local proxies retrieve online multimedia content from a distribution server over the Internet, and send it to near-by users running our mobile clients over WiFi. In this project, I take responsibility for the part of designing the application for mobile clients.}\\
% \cventry{}{[2] Challenged Content Delivery Network}{}{}{}
% {Using mobile devices to watch online multimedia content is getting increasingly popular all over the world. However, in developing countries, rural areas, or over-populated cities, many mobile users do not have Internet access. For instance, only 15\% of mobile users have mobile Internet access in Africa, because of weak or non-existing network infrastructure. Therefore, people in these areas have little chance to access online multimedia content, such as news, advertisements, and movies. That is, they suffer from digital divide from other parts of the world. We propose a Challenged Content Delivery Network (CCDN) for: (i) mobile users without stable Internet access to prefetch multimedia content, and (ii) service providers to deliver their multimedia content to more people. Our CCDN system consists of several entities: a distribution server, several local proxies, and many mobile clients. Local proxies are intelligent WiFi access points with storage spaces, which are deployed at crowded places. Local proxies retrieve online multimedia content from a distribution server over the Internet, and send it to near-by users running our mobile clients over WiFi. In this project, I take responsibility for the part of designing the application for mobile clients.}\\

