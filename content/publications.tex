% publications

% notes:
% I'd like a way to turn on/off the acceptance rates
\section{Publications}
% \subsection{Paper}
% this part copied from the bbl file
\begin{thebibliography}{1}
\bibitem{ISMIR2022}
    \bibauthor{Chih-Pin Tan}, Alvin Wen-Yu Su, Yi-Hsuan Yang.
    \newblock \emph{Melody Infilling with User-Provided Structural Context}.
    \newblock in Proc. \emph{International Conference on Music Information Retrieval}, 2022.

\bibitem{ISMIR2022LBD}
    \bibauthor{Chih-Pin Tan}, Chin-Jui Chang, Alvin Wen-Yu Su, Yi-Hsuan Yang.
    \newblock \emph{Music Score Expansion with Variable-Length Infilling}.
    \newblock in \emph{Late Breaking Demo, International Conference on Music Information Retrieval}, 2021.

\end{thebibliography}

% "Optimizing Offline Access to Social Network Content on Mobile Devices", in Proc. of IEEE INFOCOM ’14, Toronto, Canada, April 2014 \
% Y. Li, C. Trang, S. Wang, X. Huang, C. Hsu, and P. Lin, "A Resource-Constrained Asymmetric Redundancy Elimination Algorithm", Accepted to appear , IEEE/ACM Transactions on Networking, April, 2014.

% \subsection{Extended Abstract}
% 
% \begin{thebibliography}{1}
% 
% \bibitem{MM15demo}
% H. Hong, S. Wang, \bibauthor{C. Tan}, T. EI-Ganainy, K. Harras, C. Hsu, and M. Hedeeda.
% \newblock \emph{Challenged Content Delivery Network: Eliminating the Digital Divide}.
% \newblock in \emph{ Proc. of ACM Multimedia () }, Brisbane, Australia, October 2015, Demo Session.

%\bibitem{INFOCOM14}
%N. Do, Y. Zhao, \bibauthor{S. Wang}, C. Hsu, and N. Venkatasubramanian.
%\newblock \emph{{Optimizing Offline Access to Social Network Content on Mobile Devices}},
%\newblock in \emph{Proc. of IEEE INFOCOM ’14}, Toronto, Canada, April 2014.

%\bibitem{Middleware13}
%N. Do, Y. Zhao, \bibauthor{S. Wang}, C. Hsu, and N. Venkatasubramanian.
%\newblock \emph{{O2SM: Enabling Efficient Offline Access to Online Social Media and Social Networks}},
%\newblock in \emph{Proc. of ACM/IFIP/USENIX International Conference on Middleware Middleware (Middleware’13)}, Beijing, China, December 2013. 

%\bibitem{Middleware13Demo}
%N. Do, Y. Zhao, \bibauthor{S. Wang}, C. Hsu, and N. Venkatasubramanian.
%\newblock \emph{{OFacebook: Enabling Offline Access to Facebook Streams on Mobile Devices}}, Demo paper, 
%\newblock in \emph{Proc. of ACM/IFIP/USENIX International Conference on Middleware Middleware (Middleware’13)}, Beijing, China, December 2013. 

%\bibitem{Mobisys13Poster}
%\bibauthor{S. Wang}, T. Lin, Y. Wang, C. Hsu, and X. Liu.
%\newblock \emph{{Fusing Prefetch and Delay-Tolerant Transfer for Mobile Videos}}, Poster paper,
%\newblock in \emph{Proc. of ACM MobiSys ‘13}, Taipei, Taiwan, June 2013.
%\end{thebibliography}

\newpage
\section{Projects}
\cvline{}{\textbf{Multi-track Music Generation with Resampling} (on-going)}
\cvlistitem{Nowadays multi-track music-generating models generate all tracks concurrently. We are curious about if there is another generating strategy such as generating one track at a time and "modifying" this track to fit the others if necessary by Gibbs sampling.}
\cvlistitem{Acquired skills: VAE, disentanglement, Adding time-varying conditions for Transformers}
\cvlistitem{Collaborate with Hao-Wen Dong \href{https://salu133445.github.io}{(link)}, University of California San Diego.}
\cvline{}{}
\cvline{}{\textbf{Controllable Melody Infilling} (\href{https://arxiv.org/abs/2210.02829}{paper}/\href{https://github.com/tanchihpin0517/structure-aware_infilling}{code}/\href{https://tanchihpin0517.github.io/structure-aware_infilling/}{demo})}
\cvlistitem{Preserving the musical structure in the infilling process is a challenging task. Hence we try to give the model some "tips" to help it generate the melody under the given contour.}
\cvlistitem{Build the model based on TransformerXL, and fully understand the \emph{\textbf{Attention mechanism}} and the technique used in Transformer-family models including \emph{\textbf{variant token representation}}, \emph{\textbf{special positional encoding}}, and \emph{\textbf{memoriztion mechanism}}.}
\cvlistitem{Use tricky \emph{\textbf{segment embedding}} to change the TransformerXL from sequential generating model to infilling model.}
\cvlistitem{Implement an "\emph{\textbf{Attention-Selecting}}" module to receive a certain context information from the encoder}
\cvlistitem{Solve over-imitation problem by different training strategies (considering the loss of given prompts).}
\cvlistitem{Learn how to present the completed work, including \emph{\textbf{formal paper writing}}, \emph{\textbf{release of open-source code \& documentation}}, and \emph{\textbf{paper presentation}}.}
\cvlistitem{``(1) The paper is in general well written and the logic flows. (2) The authors are quite resourceful. They put just the right resources together to achieve the task. (pop909, it's structural lables, transformer, etc. (3) \emph{\textbf{adding structural control is a hard task, and to use cross-attention may be a good choice}}.'' - meta-reviwer of ISMIR 2022}
\cvline{}{}
\cvline{}{\textbf{Music Score Expansion} (\href{https://arxiv.org/abs/2111.06046}{paper}/\href{https://github.com/tanchihpin0517/variable-length-piano-expansion}{code}/\href{https://tanchihpin0517.github.io/variable-length-piano-expansion/}{demo})}
\cvlistitem{Expand 12-bars piano score to 16-bars with Transformer-based model, variable-length infilling (VLI), by \emph{\textbf{adjusting data format}}.}
\cvlistitem{Evaluate on 20 musical segments from AILabs-Pop1k7 dataset, and find the potential of music expansion with infilling models by analyzing melodic and rhythmic properties.}
\cvlistitem{``The basic idea of your late-breaking-demo, the application of "infilling" algorithms to musical prolongation, is quite interesting. \emph{\textbf{In fact, I've never quite understood the real *musical* potential of infilling, and this is an excellent demonstration}}.'' - reviwer of ISMIR 2021}

% \section{Projects}
% \cventry{}{[1] Improve Spark Performance with Auto-checkpoint}{}{}{}
% {Using mobile devices to watch online multimedia content is getting increasingly popular all over the world. However, in developing countries, rural areas, or over-populated cities, many mobile users do not have Internet access. For instance, only 15\% of mobile users have mobile Internet access in Africa, because of weak or non-existing network infrastructure. Therefore, people in these areas have little chance to access online multimedia content, such as news, advertisements, and movies. That is, they suffer from digital divide from other parts of the world. We propose a Challenged Content Delivery Network (CCDN) for: (i) mobile users without stable Internet access to prefetch multimedia content, and (ii) service providers to deliver their multimedia content to more people. Our CCDN system consists of several entities: a distribution server, several local proxies, and many mobile clients. Local proxies are intelligent WiFi access points with storage spaces, which are deployed at crowded places. Local proxies retrieve online multimedia content from a distribution server over the Internet, and send it to near-by users running our mobile clients over WiFi. In this project, I take responsibility for the part of designing the application for mobile clients.}\\
% \cventry{}{[2] Challenged Content Delivery Network}{}{}{}
% {Using mobile devices to watch online multimedia content is getting increasingly popular all over the world. However, in developing countries, rural areas, or over-populated cities, many mobile users do not have Internet access. For instance, only 15\% of mobile users have mobile Internet access in Africa, because of weak or non-existing network infrastructure. Therefore, people in these areas have little chance to access online multimedia content, such as news, advertisements, and movies. That is, they suffer from digital divide from other parts of the world. We propose a Challenged Content Delivery Network (CCDN) for: (i) mobile users without stable Internet access to prefetch multimedia content, and (ii) service providers to deliver their multimedia content to more people. Our CCDN system consists of several entities: a distribution server, several local proxies, and many mobile clients. Local proxies are intelligent WiFi access points with storage spaces, which are deployed at crowded places. Local proxies retrieve online multimedia content from a distribution server over the Internet, and send it to near-by users running our mobile clients over WiFi. In this project, I take responsibility for the part of designing the application for mobile clients.}\\

% \section{Master Thesis}
% \cventry{}{Structure-Aware Music Score Infilling via Transformer-based Models}{}{}{}
% {The purpose of this thesis is to apply music structure information to the Transformer- based models of automatic music score generation systems. In many music score gener- ation applications, we focus on music score infilling, i.e., generating a music sequence to fill in the gap between given past and future contexts. Known researches have demon- strated that prompt-based conditioning approaches could make great results in local smoothness among past context, future context, and the generated sequence. However, these cannot guarantee the repeatness and similarity corresponding to the structures of musical context. Therefore, we propose a structure-based conditioning approach, which hires a novel attention-selecting module and explicitly makes the model refer to the given structure information in the training process, on a Transformer-based model, TransformerXL, to solve the problem of the loss of structural completeness. We report on objective and subjective evaluations of the proposed models and variants of conven- tional prompt-based baselines, including comparisons of melody, rhythm and tonality, and human listening tests, to show that our approach greatly improves the generation of pop music by efficiently taking advantage of music structure information.}
% \\
% \\
% \begin{CJK}{UTF8}{bsmi}
% \cventry{}{使用 Transformer 類別深度學習模型於結構資訊相關的樂譜填空生成之應用}{}{}{}
% {本研究主要探討音樂段落資訊對於 Transformer 的衍伸模型於樂譜填寫之應用。 樂譜填寫屬於條件生成類型的問題:給予前後語境,生成一串音樂序列來填補中間的空缺。在已知關於音樂填寫的研究中,對於生成結果局部性順暢的問題已經有了系統性的解決方法。 然而,基於提示的條件狀況沒辦法保證生成結果的結構性以及相似性,換句話說,生成結果是否符合音樂整體的架構以及生成的結果是否與對應的音樂段落相似。 因此,在本研究中,我們提出了基於條件狀況的方法,藉由在訓練過程中明確的要求模型參考所提供的段落資訊,並增加了全新的注意力選擇模組於 Transformer 的衍生模型,讓音樂生成的過程可以更有效率地使用使用者提供的段落資訊提示,來解決基於提示條件狀況的音樂生成結果中所缺失的結構性和相似性的問題。 在實驗方法中,我們與其他人的研究結果做比較,衡量方式包含旋律,節奏與調性的相似度計算,以及志願者的聽力測試,來顯示我們所提出的新架構能更有效率地利用音樂段落資訊。}
% \end{CJK}

% \begin{CJK}{UTF8}{bsmi}
% \section{Side Projects}
% \cventry{}{使用Rust開發四軸飛行器 (進行中)}{}{}{}
% {使用Rust開發四軸飛行器,並使用Rust開發相關的控制程式。
% 開發硬體為Respberry Pi Pico,目標為以Rust為開發語言,設計簡易的OS以及所需的driver,OS包含thread和timer並提供基本的system call,而驅動的部分包含藍芽以及馬達,涉及的範圍包含GPIO、PWM、UART等韌體相關程式撰寫。}
% \end{CJK}